\chapter*{}

\section*{Resumen (español)}
Este Trabajo de Fin de Grado tiene principalmente como objetivo la ampliación de las capacidades de análisis de las herramientas para la verificación de los sistemas ciberfísicos en tiempo de ejecución. Los requisitos de la verificación están expresados en  Signal Temporal Logic (STL), un tipo de lógica temporal enfocada al análisis de señales analógicas o reales. La lógica temporal es su vez un tipo de lógica modal que nos permite expresar propiedades sobre un estado o una secuencia de estados del sistema. La ampliación consiste en añadir dos nuevos operadores lógicos, la derivada (tendencias) y la integral (acumulaciones).

La segunda parte del proyecto consiste en probar el buen funcionamiento de una biblioteca de minería de datos ya existente para casos paramétricos con los nuevos operadores y ofrecer una nueva interfaz gráfica para una interacción mas fácil tanto con STL como con la biblioteca de minería.

\textbf{Palabras Clave}:
\begin{itemize}
\item STL
\item Lógica Temporal
\item C++
\item Python
\item Interfaz Gráfica
\item PyQt
\item Minería de Datos 
\end{itemize}
\newpage

\section*{Abstract (English)}
\begin{otherlanguage}{british}
The purpose of this Final Degree Project is mainly to expand the analysis capabilities of the tools for the verification of cyber-physical systems at runtime. The verification requirements are expressed in Signal Temporal Logic (STL), a type of temporal logic focused on the analysis of analog or real signals. Temporal logic is in turn a type of modal logic that allows us to express properties about a state or a sequence of states of the system. The extension consists of adding two new logical operators, the derivative (trends) and the integral (accumulation).

The second part of the project consists of testing the functionality of an existing data mining library for parametric cases with the new operators and offering a new graphical interface for easier interaction with both STL and the mining library.
\end{otherlanguage}

\textbf{Keywords}:
\begin{itemize}
\item STL
\item Temporal Logic
\item C++
\item Python
\item GUI
\item PyQt
\item Data Mining
\end{itemize}