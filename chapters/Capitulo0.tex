\chapter*{}

\section*{Resumen}
Este Trabajo de Fin de Grado tiene principalmente como objetivo la ampliación de las capacidades de análisis de las herramientas para la verificación de los sistemas ciberfísicos en tiempo de ejecución. Los requisitos de la verificación están expresados en  Signal Temporal Logic (STL), un tipo de lógica temporal enfocada al análisis de señales analógicas o reales. La lógica temporal es un tipo de lógica modal que nos permite expresar propiedades sobre un estado o una secuencia de estados del sistema. La primera mejora consiste en añadir dos nuevos operadores lógicos para definir tendencias (derivadas) y acumulaciones (integrales).

%La segunda parte del proyecto consiste en probar el buen funcionamiento de una biblioteca de minería de datos ya existente para casos paramétricos con los nuevos operadores y ofrecer una nueva interfaz gráfica para una interacción mas fácil tanto con STL como con la biblioteca de minería.

La segunda parte del proyecto consiste en incorporar los nuevos operadores lógicos en una una biblioteca de minería de datos ya existente para evaluar expresiones de STL paramétrica. Hemos comprobado el buen funcionamiento de los nuevos operadores y hemos diseñado una nueva interfaz gráfica para ofrecer una interacción mas fácil tanto con STL como con la biblioteca de minería.

\textbf{Palabras Clave}:
\begin{itemize}
\item Lógica Temporal
\item STL
\item Derivada
\item Integral
\item Interfaz Gráfica
\item PyQt
\item Minería de Datos 
\item C++
\item Python
\end{itemize}
\newpage

\section*{Abstract}
\begin{otherlanguage}{british}
The purpose of this Final Degree Project is mainly to expand the analysis capabilities of the tools for the verification of cyber-physical systems at runtime. The verification requirements are expressed in Signal Temporal Logic (STL), a type of temporal logic focused on the analysis of analog or real signals. Temporal logic is a type of modal logic that allows us to express properties about a state or a sequence of states of the system. The first improvement consists of adding two new logical operators for defining trends (derivatives) and accumulations (integrals).

The second part of the project consists of including the new logical operators to an existing data mining library for evaluating parametric STL expressions. We have tested the functionality of the new operators and designed a new graphical interface that offers an easier interaction with both STL and the mining library.
\end{otherlanguage}

\textbf{Keywords}:
\begin{itemize}
\item Temporal Logic
\item STL
\item Derivative
\item Integral
\item GUI
\item PyQt
\item C++
\item Python
\item Data Mining
\end{itemize}