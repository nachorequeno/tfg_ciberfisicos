\chapter{Conclusiones}
\label{cha:concl}
Para acabar este trabajo presentamos algunas conclusiones sobre los temas tratados, tanto teóricos como prácticos, y discutimos algunas ideas propicias para la mejora y continuidad del proyecto.
\section{Conclusiones (español)}
En vista a los resultados del capítulo anterior $\ldots$

\section{Trabajo futuro}

Incluir más opciones en la GUI que nos permitan configurar parámetros adicionales de ParetoLib: p.ej., (des)activar el paralelismo, precisión del aprendizaje (EPS, DELTA, STEPS) ...

Extender STLe para soportar nuevos tipos de interpolación (lineal, polinómica, etc.) Actualmente sólo soporta interpolación constante.

Implementar un procesador de lenguaje natural que facilite la escritura de expresiones en STL en un formato más agradable. P. ej.:
\begin{itemize}
 \item ``En el futuro, la propiedad X se cumple''
 \item ``F G (v > 120)'' en lugar de ``(F (G ( > v 120) )''
\end{itemize}

Reimplementar el núcleo de ParetoLib para aumentar las prestaciones computacionales: Python se puede compilar en lugar de interpretar, lo que aumentaría el rendimiento de la librería de minería. La mejora sería mínima (ParetoLib llama a STLeval para la mayoría de los cálculos, y STLeval está en C++), pero la mejora del rendiento al compilar Python nos ayudaría ahorrar unos pocos milisegundos.

\section{Conclusions (English)}
In view of the results of the previous chapter, $\ldots$


