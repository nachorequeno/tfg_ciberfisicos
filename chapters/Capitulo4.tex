\chapter{Conclusiones}
\label{cha:concl}
Para acabar este trabajo presentamos algunas conclusiones sobre los temas tratados, tanto teóricos como prácticos, y discutimos algunas ideas propicias para la mejora y continuidad del proyecto.
\section{Conclusiones (español)}

% Contribuciones
%En este proyecto, hemos actualizado los binarios y librerías dinámicas de STLEval que incorpora ParetoLib para dar soporte a las nuevas operaciones de derivación e integración. Además, hemos implementado una interfaz gráfica que abstrae la complejidad de invocar al código fuente ~\ref{list:paretolib_example} y resume la mayor parte de las opciones de configuración del algoritmo de minería.
Este TFG tiene como objetivo mejorar, diseñar, extender e implementar las herramientas STEval y ParetoLib para el tratamiento de señales aplicando la lógica temporal.

Durante el estudio realizado antes de trabajar en la extensión de las herramientas, hemos podido comprobar el arduo trabajo que llevan a cabo los investigadores de todas las materias para poder conseguir resultados y/o conclusiones que presentar, esta experiencia nos ha permitido valorar mucho más de lo que creíamos el esfuerzo y la labor de investigación de todos los profesionales, además de darnos  la satisfacción de poder tener este papel de investigadores en una versión más minúscula pero suficientemente retadora para nosotros. 

Como parte ya de la resolución del proyecto llegamos a extender las capacidades de la lógica temporal en la herramienta STLEval añadiendo las propiedades de la derivada, esto nos permitió comprender cómo  la recogida de datos unida a esta función matemática es capaz de mostrarnos tendencias y éstas son las que, actualmente, estamos viendo que van ganando más terreno tanto en los negocios como en nuestras vidas; y por otra parte la integral que se encarga de ... .Desde la aplicación creada se puede dar uso a éstas operaciones sin tener que pasar por la consola como en un principio ocurría.

Por otro lado, después de encargarnos de la implementación, aprendimos cómo conectar 2 herramientas que no estaban escritas en el mismo lenguaje de programación, esta tarea aunque al principio bastante complicada debido a que nunca antes la realizamos, la pudimos completar gracias a la ayuda de nuestro tutor quien siempre nos ofreció su ayuda, muchas veces incluso fuera del horario laboral. Además de esta integración actualizamos los binarios y librerías dinámicas de STLEval que incorpora ParetoLib para dar soporte a las nuevas operaciones de derivación e integración. 

Nuestra aplicación no dispone de una base de datos al trabajar con la propia y diferente información que va a tratar el usuario, es responsabilidad suya poseerla.

Para finalizar hemos implementado una interfaz gráfica que abstrae la complejidad de invocar al código fuente ~\ref{list:paretolib_example} y resume la mayor parte de las opciones de configuración del algoritmo de minería.


\section{Trabajo futuro}

Incluir más opciones en la GUI que nos permitan configurar parámetros adicionales de ParetoLib: p.ej., (des)activar el paralelismo, precisión del aprendizaje (EPS, DELTA, STEPS) ...

Actualmente STLEval sólo soporta interpolación constante. Como trabajo futuro, se plantea extender dicha herramienta para soportar nuevos tipos de interpolación (lineal, splines, etc.). Esto abre la posibilidad de desarrollar nativamente nuevos tipos de operadores lógico-temporales que permitan realizar predicciones sobre el futuro en base a unas señales ya conocidas y recogidas del artefacto que queramos monitorizar (aproximación mediante series de Taylor, análisis estadístico de las trazas u otros métodos de aprendizaje).

Implementar un procesador de lenguaje natural que facilite la escritura de expresiones en STL en un formato más agradable. P. ej.:
\begin{itemize}
 \item ``En el futuro, la propiedad X se cumple''
 \item ``F G (v > 120)'' en lugar de ``(F (G ( > v 120) )''
\end{itemize}

Reimplementar el núcleo de ParetoLib para aumentar las prestaciones computacionales: Python se puede compilar en lugar de interpretar, lo que aumentaría el rendimiento de la librería de minería. La mejora sería mínima (ParetoLib llama a STLeval para la mayoría de los cálculos, y STLeval está en C++), pero la mejora del rendiento al compilar Python nos ayudaría ahorrar unos pocos milisegundos.

\section{Conclusions (English)}
In view of the results of the previous chapter, $\ldots$


