%\chapter{Introducción histórica}
%
%\section{Antecedentes históricos y actualidad (español)}
%Primera sección.
%
%\section{Historical and current background (English)}
%
%\begin{otherlanguage}{british}
%En inglés
%\end{otherlanguage}

\chapter{Introducción}

\section{Sistemas ciberfísicos}
Los sistemas ciberfísicos son un tipo especial de sistemas en tiempo real que combinan un micro controlador o programa software, representado mediante una máquina de estados discretos, con uno o varios sensores que interactúan sobre una variable física. Ejemplos de este tipo de entornos son los ordenadores de a bordo que regulan la velocidad de crucero de un vehículo, o el piloto automático que controla la altitud y trayectoria de un avión.

Garantizar el correcto funcionamiento de estas plataformas es crucial, ya que un mal funcionamiento conlleva una reducción del comfort por parte de los usuario o, incluso, la pérdida de vidas humanas. Una manera de asegurarlo es mediante el uso de técnicas de verificación formal. Las técnicas de verificación en tiempo de ejecución \cite{STTT_RV_21} (\emph{runtime verification} en inglés) implementan un monitor que supervisa que la ejecución actual cumple con los requisitos especificados. Los monitores actúan como guardianes, alertando al usuario cuando la ejecución se desvía del comportamiento deseado e, incluso, implementando contramedidas para revertir ese hecho y redirigir el sistema hacia un estado saludable. 

Existen diversos formalismos para definir los comportamientos deseados y no deseados. Los requisitos funcionales se pueden expresar operacionalmente, mediante algún tipo de autómata de estados finito, o declarativamente, mediante una descripción lógico-matemática. Dentro de esta segunda categoría, la lógica temporal es un tipo de lógica modal que expresa propiedades sobre un \emph{estado} en particular del sistema, o sobre los \emph{caminos} (es decir, secuencia de estados) que atraviesa. En este trabajo utilizaremos Signal Temporal Logic \cite{STL}, un tipo de lógica temporal enfocada al análisis de señales analógicas.

\section{Objetivos}

Los objetivos de este proyecto son:

\begin{itemize}
\item Extender las capacidades de la lógica temporal para expresar propiedades que involucren \textit{tendencias} (derivadas), o \textit{acumulaciones} (integrales), mediante el cálculo de éstas nuevas operaciones podemos realizar operaciones de predicción a futuro en base a unas señales ya conocidas y recogidas del artefacto que querramos monitorizar.   
\item Implementar esos nuevos operadores lógicos en las herramientas software actuales con el fin de extender su funcionamiento no sólo aplicado a todo un conjunto de señales sino también a un espacio de tiempo en particular que se encuentre dentro de nuestro conjunto principal. 
\item Y por último, proporcionar una interfaz de usuario amigable que facilite la interacción con dichas herramientas, para este apartado  proponemos un espacio donde el usuario tenga que simplemente adjuntar los fichero de datos y operaciones (o elegir la operación a realizar desde las opciones disponibles) para finalmente poder visualizar y descargar la salida de señales en forma gráfica. 
\end{itemize}

\begin{document}

\section{Organización}

\subsection{Documento}

El documento está dividido en X capítulos. En el capítulo \ref{cha:stl} detallaremos la semántica e implementación de los nuevos operadores lógicos. El capítulo \ref{cha:gui} está dedicado a la nueva interfaz gráfica de usuario. Continuamos con las conclusiones más relevantes del proyecto en el capítulo \ref{cha:concl}. Finalmente $\ldots$.

\subsection{Equipo de trabajo}

Este proyecto ha sido realizado por dos estudiantes, Dymtro ap1 ap2 y Javier ap1 ap2, acompañados del tutor Jose Ignacio Requeno. Se empezó a realizar el X de septiembre del 2021 y constó de 4 fases: 

\subsubsection{Primera - Investigación}

	En esta fase se inició la investigación del proyecto, nos dedicamos a leer información sobre qué es la lógica temporal y la diferencia entre lógica modal, recurrimos tanto al material proporcionado por el instructor como a material externo y aprendimos un poco de su sintáxis. Esta fué la parte más complicada de todo el trabajo no sólo por la poca información que éxiste de este tema en la red, obligándonos a hacer búsquedas más exhaustivas y profundas sobre los datos que teníamos sino también por el cambio de perspectiva que tuvimos que realizar para comprender el concepto principal de ésta lógica que es que: Una afirmación se puede convertir en negación dependiendo del tiempo en el que se evalúa la expresión. 
	
\subsubsection{Segunda - Documentación}

	Después del proceso de cambio empezamos con el de lectura y comprensión de la librería, su arquitectura y el significado de las clases además del estudio teórico de cómo funciona una integral y una derivada y cómo podemos implementarlo mediante código, en esta fase nuestro mayor dificultad fué la labor de documentación del código e interpretación del conocimiento adquirido en el apartado anterior trasladado a éste. 

\subsubsection{Tercera - Implementación}
	
	Realizamos la implementación de la integral y derivada extendiendo la librería STLe para que pueda ser aplicada a un conjunto y subconjunto de señales, para completar esta tarea realizamos una serie de pruebas con señales conocidas que nos proporcionó el tutor además de pruebas individuales realizadas con datos aleatorios y comprobando matematicamente sus resultados para comprobar que ambos, la salida de lo que había sido programado y lo que habíamos realizado a mano, sean similares.

\subsubsection{Cuarta - Diseño y Finalización}

	Como última tarea creamos la interfáz gráfica de la aplicación realizando un boceto que luego intentamos trasladar a la aplicación, la tarea más complicada de esta parte fué la realización de la conexión entre los diferentes layouts de las clases de la interfaz y los métodos propios de la extensión de la libreria STLe, nos ayudó bastante el hecho de que la API ya estaba realizada por parte de la librería "multidimensional\_search", al decidir codificar está parte de la aplicación con el lenguaje Python nos resulto mucho más facil las tareas de tratamiento de datos y creemos que es un lenguaje que puede facilitar la implementación de las futuras operaciones que sugeriremos al final de este documento. 


\subsection{Planificación y Organización}

Debido a la dificultad matemática de este proyecto para nosotros, decidimos realizar las tareas de manera conjunta, todos los apartados de este proyecto excepto la labor de documentación se realizaron a la par por ambos integrantes del grupo. 

La planificación coincide con las fases del proyecto, indicamos las fechas exactas en el siguiente gráfico: 

Diagrama Gant **GRAFICO**

\end{document}